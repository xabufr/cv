\documentclass[11pt,a4paper]{moderncv}
\moderncvtheme[blue]{classic}                
\usepackage[utf8]{inputenc}
\usepackage{textcomp}
\usepackage[top=1.0cm, bottom=1.0cm, left=1.0cm, right=1.0cm]{geometry}
 
\usepackage[french]{babel}     % prend en charge les spécificités du français
 
\usepackage{lmodern}           % charge les polices LatinModern appropriées pour un rendu PDF

\setlength{\hintscolumnwidth}{2.5cm}

\usepackage{datenumber,fp}

\newcounter{dateofbirth}
\newcounter{dateoftoday}

\linespread{1.1}

\setmydatenumber{dateofbirth}{1992}{07}{02}
\setmydatenumber{dateoftoday}{\the\year}{\the\month}{\the\day}
\FPsub\result{\thedateoftoday}{\thedateofbirth}
\FPdiv\myage{\result}{365.2425}

\firstname{Thomas}
\familyname{Loubiou}
\title{{\Large Développeur Backend, disponible janvier 2020}}
\address{1 Allée Marie Louise}{}{92290 Chatenay-Malabry}
\email{thomas.loubiou@yahoo.com}
\social[github]{xabufr}
\social[linkedin]{thomas-loubiou}
\mobile{06 60 91 00 94}
\extrainfo{\FPtrunc\myage{\myage}{0}\myage\ ans --- Permis A \& B}
\photo[64pt][0.4pt]{photo}

\begin{document}
\makecvtitle%
\vspace*{-1.5\baselineskip}
\section{Expériences Professionnelles}
\cventry{2018--Aujourd'hui}{Directeur Technique}{Zyps}{}{}{
  \begin{itemize}
  \item Mise en place de l'infrastructure -- {\tiny Ansible, Ubuntu, Kubernetes on BareMetal, Prometheus, Grafana }
  \item Architecture -- {\tiny Keycloak, RabbitMQ, ElasticSearch, Postgres, Micro-services, S3-like }
  \item Développement backend -- {\tiny Rust, Node.js/Typescript, Python }
  \end{itemize}
}
\cventry{2018--2019}{Développeur fullstack}{Corporama}{}{}{Développement et maintenance de Corporama -- {\tiny Erlang/OTP, jQuery }}
\cventry{2015--2018}{Développeur, Responsable recherche et développement}{Systonic}{}{KeepAlert}{Missions diverses
\begin{itemize}
    \item Amélioration des outils internes {\tiny -- Java}
    \item Début de moteur de recherche d'image (CBIR) et autres missions de R\&D {\tiny -- C++, Python},
    \item Écriture d'un moteur de Web-Scraping, création d'un pseudo-langage descriptif dédié {\tiny -- TypeScript},
    \item Création de la version 2 de la plateforme KeepAlert {\tiny -- Java, Kotlin, Spring Boot, Cloud AWS, Kubernetes}
\end{itemize}
}
\cventry{2013--2015}{Développeur en alternance}{Systonic}{}{KeepAlert \& Prodomaines}{Réalisation de nombreux projets\@:
\begin{itemize}
    \item Rédaction d'un article pour ``Programmez!'' (\textnumero170--171) sur l'utilisation des annotations et de l'introspection en Java,
    \item Développement d'une solution basée sur de l'OCR {\tiny -- Bash},
    \item Modernisation des outils utilisés par l'équipe de développement (SVN vers Git, Ant vers Maven, Jenkins),
    \item Migration de la plateforme KeepAlert dans le Cloud d'Amazon {\tiny -- Java, Hadoop, ElasticSearch},
    \item Développement d'API RESTful pour les extranets KeepAlert \& Prodomaines {\tiny Java, DropWizard},
\end{itemize}
}
\cventry{2012}{Développeur Web}{Goûts de Web}{Bordeaux}{Stage 7 semaines}{Développement d'un site web de petites annonces -- {\tiny \url{marseille-culture13.fr}, PHP, Symphony 2}}
\cventry{2010}{Développeur}{EPSI}{Bordeaux}{Stage 7 semaines}{Développement d'une application de gestion des absences et des notes pour les enseignants {\tiny -- PHP, JavaScript}}
\section{Compétences principales}
\cvitem{Langages}{Java, Kotlin, Rust, Erlang, Python 3, JavaScript, TypeScript, C++, HTML/CSS, SQL}
\cvitem{Divers}{Kubernetes, Docker, Git, AWS, NodeJS, ElasticSearch, Jenkins, RabbitMQ, Spacemacs}
\cvitem{O.S.}{Gentoo \& Funtoo, ArchLinux, Debian, CentOS, Ubuntu}
\cvitem{Pratiques}{TDD, Clean Code}
\cvitemwithcomment{Anglais}{\mdseries Documentation technique, lu, écrit, TOEIC}{Bonne compréhension à l'écrit}
\section{Formation}
\cventry{2012--2015}{EPSI}{Bordeaux}{}{Expert en Informatique et Systèmes d'Information}{}
\cventry{2012}{BTS Informatique de Gestion}{EPSI Bordeaux}{}{}{}
\section{Centres d'intérêt}
\cvitem{Sports}{Escalade, Randonnées}
\cvitem{Lectures}{Veille technologique, romans d'heroic fantasy, ouvrages informatiques}
\cvitem{Informatique}{Jeux vidéo, programmation}
\end{document}

