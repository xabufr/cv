\documentclass[11pt,a4paper]{moderncv}
\moderncvtheme[blue]{classic}                
\usepackage[utf8]{inputenc}
\usepackage{textcomp}
\usepackage[top=1.0cm, bottom=1.0cm, left=1.0cm, right=1.0cm]{geometry}
 
\usepackage[french]{babel}     % prend en charge les spécificités du français
 
\usepackage{lmodern}           % charge les polices LatinModern appropriées pour un rendu PDF

\setlength{\hintscolumnwidth}{2.5cm}

\usepackage{datenumber,fp}

\newcounter{dateofbirth}
\newcounter{dateoftoday}

\setmydatenumber{dateofbirth}{1992}{07}{02}
\setmydatenumber{dateoftoday}{\the\year}{\the\month}{\the\day}
\FPsub\result{\thedateoftoday}{\thedateofbirth}
\FPdiv\myage{\result}{365.2425}

\firstname{Thomas}
\familyname{Loubiou}
\title{Candidature poste de Développeur}
\address{Clos du Mas la Senia}{Rue Albert Saisset}{66700 Argelès-sur-mer}
\email{thomas.loubiou@yahoo.com}
\social[github]{xabufr}
\mobile{06 60 91 00 94}
\extrainfo{\FPtrunc\myage{\myage}{0}\myage\ ans --- Permis A2/B}
\photo[64pt][0.4pt]{photo}
\begin{document}
\makecvtitle%
\section{Formation}
\cventry{2012--2015}{Cycle d'Ingénierie Informatique}{EPSI Bordeaux}{}{}{Option Génie Logiciel, diplôme d'Expert en Informatique et Systèmes d'Information}
\cventry{2012}{BTS Informatique de Gestion}{EPSI Bordeaux}{}{}{Option Développeur d'applications}
\cvitemwithcomment{Anglais}{Documentation technique, lu, écrit, TOEIC}{Bonne compréhension à l'écrit}
\section{Compétences principales}
\cvitem{C}{SDL, GTK, liblua}
\cvitem{C++14}{Qt4/5, Boost, SFML, Ogre3D, Bullet Physics, Box2D, FMODEx, tolua++, CEGUI, OpenCV}
\cvitem{Java}{Version 8, Spring (Boot, Cloud), Jackson, LiquiBase, Flyway, Hadoop, JUnit, Reactor}
\cvitem{Autres langages}{Python 3, ECMAScript, TypeScript, HTML, CSS, SQL, Kotlin, notions de Clojure \& Haskell, WebComponents}
% Supprimé pour plus de concision
% OAuth2, JWT, JSON, XML, Angular, PIXI, PHP, Symfony 1 & 2, SOAP, REST
%\cvitem{Cloud AWS}{EC2, CloudFormation, S3, IAM, EMR, EBS, OpsWork, SNS, CloudWatch}
\cvitem{Divers}{Vim, NeoVim, Spacemacs, bash, zsh, Eclipse, IntelliJ IDEA, Make, GCC, Clang, Gulp, Maven, NodeJS, PhantomJS, Electron, QtCreator, KDevelop, Git, CMake, Docker, Kubernetes, Latex, Pandoc, ElasticSearch, MySQL, Postgresql, Redis, SSH, Jenkins, RabbitMQ, Cloud AWS}
\cvitem{O.S.}{Gentoo \& Funtoo, ArchLinux, Debian, Fedora, Mandriva, CentOS, Ubuntu}
\cvitem{Pratiques}{TDD, Clean Code, UML, Merise}
\section{Expériences Professionnelles}
\cventry{2015--2018}{Développeur, Responsable recherche et développement}{Systonic}{}{KeepAlert}{Missions diverses
\begin{itemize}
    \item Amélioration des outils internes
    %\item Missions de R\&D diverses
    \item Début de moteur de recherche d'image (CBIR) et autres missions de R\&D,
    \item Création d'un langage de script pour le web-scraping
    \item Création de la version 2 de la plateforme KeepAlert
\end{itemize}
}
\cventry{2013--2015}{Développeur en alternance}{Systonic}{}{KeepAlert \& Prodomaines}{Réalisation de nombreux projets\@:
\begin{itemize}
    \item Rédaction d'un article pour ``Programmez!'' (\textnumero170--171) sur l'utilisation des annotations et de l'introspection en Java,
    \item Développement d'une solution basée sur de l'OCR,
    \item Modernisation des outils utilisés par l'équipe de développement (SVN vers Git, Ant vers Maven, Jenkins),
    \item Migration de la plateforme KeepAlert dans le Cloud d'Amazon (Hadoop, ElasticSearch),
    %\item Mise en place de système d'intégration continue basé sur Jenkins,
    \item Développement d'API RESTful pour les extranets KeepAlert \& Prodomaines,
\end{itemize}
}
\cventry{2012--7 sem.}{Développeur Web}{Goûts de Web}{Bordeaux}{}{Développement d'un site web de petites annonces --- \url{marseille-culture13.fr}}
\cventry{2010--7 sem.}{Développeur}{EPSI Bordeaux}{}{}{Développement d'une application de gestion des absences et des notes pour les enseignants}
\section{Centres d'intérêt}
\cvitem{Sports}{Planche à voile, VTT, Vélo, Escalade, Randonnées}
\cvitem{Lectures}{Veille technologique, romans d'heroic fantasy, ouvrages informatiques}
\cvitem{Informatique}{Jeux vidéo, programmation}
\end{document}

