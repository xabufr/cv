\documentclass[11pt,a4paper]{moderncv}
\moderncvtheme[blue]{classic}                
\usepackage[utf8]{inputenc}
\usepackage{textcomp}
\usepackage[top=1.0cm, bottom=1.0cm, left=1.0cm, right=1.0cm]{geometry}
 
\usepackage[french]{babel}     % prend en charge les spécificités du français
 
\usepackage{lmodern}           % charge les polices LatinModern appropriées pour un rendu PDF

\setlength{\hintscolumnwidth}{2.5cm}

\usepackage{datenumber,fp}

\newcounter{dateofbirth}
\newcounter{dateoftoday}

\linespread{1.1}

\setmydatenumber{dateofbirth}{1992}{07}{02}
\setmydatenumber{dateoftoday}{\the\year}{\the\month}{\the\day}
\FPsub\result{\thedateoftoday}{\thedateofbirth}
\FPdiv\myage{\result}{365.2425}

\firstname{Thomas}
\familyname{Loubiou}
\title{{\Large Développeur Backend, disponible janvier 2020}}
\address{40 rue poujeau}{}{33200 Bordeaux Caudéran}
\email{thomas.loubiou@yahoo.com}
\social[github]{xabufr}
\social[linkedin]{thomas-loubiou}
\mobile{06 60 91 00 94}
\extrainfo{\FPtrunc\myage{\myage}{0}\myage\ ans --- Permis A \& B}
\photo[64pt][0.4pt]{photo}

\begin{document}
\makecvtitle%
\vspace*{-1.5\baselineskip}
\section{Expériences Professionnelles}
\cventry{2018--Aujourd'hui}{Directeur Technique}{Zyps}{}{}{
  \begin{itemize}
  \item Mise en place de l'infrastructure -- {\tiny Ansible, Ubuntu, Kubernetes on BareMetal, Prometheus, Grafana, Wireguard, Jenkins, Gitea, Confluence, Jira, LDAP }
  \item Architecture -- {\tiny Keycloak, RabbitMQ, ElasticSearch, Postgres, Micro-services, S3-like }
  \item Développement backend -- {\tiny Rust, Node.js/Typescript, Python }
  \end{itemize}
}
\cventry{2018--2019}{Développeur fullstack}{Corporama}{}{}{Développement et maintenance de Corporama -- {\tiny Erlang/OTP, jQuery, Peer-review }
  \begin{itemize}
  \item Développement de nouvelles fonctionnalités, maintenance et correction d'anomalies, support client
  \item Début de mise en place de machine learning/NLP avec Spacy
  \end{itemize}
}
\cventry{2015--2018}{Développeur, Responsable recherche et développement}{Systonic}{}{KeepAlert -- Plateforme SaaS de surveillance de marques sur Internet}{Missions diverses, réalisées seul et en autonomie
\begin{itemize}
    \item Amélioration et maintenance d'un outil interne de facturation {\tiny -- Java, Everwin}
    \item R\&D sur un moteur de recherche d'image (Content Based Image Retrieving) orienté logo {\tiny -- C++/OpenCV, Python},
    \item R\&D sur les patterns de noms de domaines pour améliorer les taux de détection de KeepAlert (tentative de publication d'un papier de recherche dans une revue scientifique) {\tiny Hadoop, C++},
      {\item Création d'un moteur de web-scraping et d'une plateforme SaaS associée
        --
        Utilisé par KeepAlert pour la récupération de données sur les réseaux sociaux, les boutiques en ligne, les moteurs de recherche et tout autre site web ne fournissant pas d'API satisfaisante.
        {\tiny Typescript, NodeJS}
        \begin{itemize}
        \item Gestion des sites dynamiques, des évènements asynchrones, du parallélisme, des captchas\ldots
        \item Gestion de plusieurs navigateurs (PhantomJS/SlimerJS, Electron, Google Chrome), 
        \item Création d'un langage dédié pour faciliter l'écriture des scénarios de scraping basé sur JSON/Javascript,
        \item Création d'une interface d'édition et de test de scripts avec Electron,
          {\item Création d'une plateforme SaaS avec versioning des scénarios de scraping, gestion des récurrences pour des lancements programmés, des notifications via webhooks,
            API documentée avec Swagger, interface web en Polymer 1.0, authentification OAuth2/JWT {\tiny AWS, AWS EC2, AWS Lambda, Kubernetes, beanstalkd, Java/Spring Boot, NodeJS, Typescript, Polymer, S3}}
        \end{itemize}
      }
      {\item Création de la version 2 de la plate-forme KeepAlert.
        \\
        KeepAlert étant trop vieillissant en terme d'interface et de capacités de récupération des données une refonte complète a été entamée.
        Bien que menée à bien et fonctionnelle en interne, cette nouvelle version n'a jamais été publiée par manque de moyens.
        {\tiny -- Java 8, Kotlin, RabbitMQ, Spring Boot, Sprint Integration, Cloud AWS (S3, EC2, VPC, ElasticIP, LoadBalancer, EKS, ECR), Docker, Kubernetes, Elasticsearch, Postgresql, Jenkins/fabric8, Grafana, Prometheus}
        \begin{itemize}
        \item Architecture en micro-services pour faciliter la répartition des charges au sein d'un cluster et utiliser les technologies les plus adaptées en fonction des services,
        \item Utilisation de RabbitMQ pour une communication inter-service fiable et asynchrone,
        \item Utilisation de Kubernetes via kube-aws puis KOPS et spot instances sur AWS EC2,
        \item Mise en place d'Elasticsearch 2.x sur un cluster Kubernetes,
        \item Mise en place de centralisation des logs sur Kubernetes,
        \item Mise en place de l'intégration/déploiement continu grâce à Jenkins et Fabric8,
        \item Écriture des micro-services en Java 8/Kotlin/Spring Boot/Spring Integration/Spring Security, Ruby, Python, Test-Driven Developpment,
        \item Intégration des services avec Zipkin (distributed tracing),
        \item Sécurisation OAuth/JWT,
        \item Encadrement du stagiaire développant l'interface web,
        \end{itemize}
      }
\end{itemize}
}
\cventry{2013--2015}{Développeur en alternance}{Systonic}{}{KeepAlert \& Prodomaines}{Réalisation de nombreux projets\@:
\begin{itemize}
    \item Rédaction d'un article pour ``Programmez!'' (\textnumero170--171) sur l'utilisation des annotations et de l'introspection en Java pour la génération de documents Excel à partir d'un modèle Java,
    \item Développement d'une solution OCR pour extraction automatisée de données textuelles {\tiny -- Bash, gOCR},
    \item R\&D pour un système de classification de domaines/site web avec machine learning en Java 7/Mallet,
    \item Amélioration du parsing des données Whois grâce à la bibliothèque ruby whois-parser et JRuby pour l'utiliser avec le code existant Java 7,
    \item Amélioration du système de capture d'écran de site web grâce au navigateur headless PhantomJS,
    \item Modernisation des outils utilisés par l'équipe de développement (SVN vers Git, Ant vers Maven, intégration continue avec Jenkins),
    \item Migration de la production des études KeepAlert dans le Cloud d'Amazon grâce à Hadoop pour en accélérer la production, améliorer la fiabilité et permettre de récolter plus de données {\tiny -- AWS EMR, Java, Hadoop, ElasticSearch},
    \item Écriture d'un service dédié à l'orchestration des clusters EMR/Hadoop sur AWS afin d'en optimiser l'utilisation et les coûts {\tiny Java/DropWizard/MySQL/Angular 1},
    \item Passage d'un stockage de données SQL vers un stockage Elasticsearch 1.X afin d'augmenter le nombre de champs disponibles ainsi que les possibilités de recherche,
    \item Développement d'API RESTful accessibles en lecture pour les extranets KeepAlert \& Prodomaines {\tiny Java, DropWizard},
\end{itemize}
}
\cventry{2012}{Développeur Web}{Goûts de Web}{Bordeaux}{Stage 7 semaines}{Développement d'un site web de petites annonces -- {\tiny \url{marseille-culture13.fr}, PHP, Symphony 2}}
\cventry{2010}{Développeur}{EPSI}{Bordeaux}{Stage 7 semaines}{Développement d'une application de gestion des absences et des notes pour les enseignants {\tiny -- PHP, JavaScript}}
\section{Compétences principales}
\cvitem{Langages}{Java, Kotlin, Rust, Erlang, Python 3, JavaScript, TypeScript, C++, HTML/CSS, SQL}
\cvitem{Divers}{Kubernetes, Docker, Git, AWS, NodeJS, ElasticSearch, Jenkins, RabbitMQ, Spacemacs}
\cvitem{O.S.}{Gentoo \& Funtoo, ArchLinux, Debian, CentOS, Ubuntu}
\cvitem{Pratiques}{TDD, Clean Code}
\cvitemwithcomment{Anglais}{\mdseries Documentation technique, lu, écrit, TOEIC}{Bonne compréhension à l'écrit}
\section{Formation}
\cventry{2012--2015}{EPSI}{Bordeaux}{}{Expert en Informatique et Systèmes d'Information}{}
\cventry{2012}{BTS Informatique de Gestion}{EPSI Bordeaux}{}{}{}
\section{Centres d'intérêt}
\cvitem{Sports}{Escalade, Randonnées}
\cvitem{Lectures}{Veille technologique, romans d'heroic fantasy, ouvrages informatiques}
\cvitem{Informatique}{Jeux vidéo, programmation}
\end{document}

